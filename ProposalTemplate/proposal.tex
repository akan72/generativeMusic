\documentclass{article}
\usepackage{fullpage}
\title{A novel model of prey behavior}
\author{Wille E. Coyote}
\begin{document}
\maketitle
\paragraph{Problem and its relevance to health}
High mortality among Canis latrans (coyote) involved in discovery, chase and capture of Geococcyx Californianus (road runner) and Lepus timidus
 (hare) can be attributed to poor modeling of the natural movement patterns of the prey. Better understanding movement and evasion patterns of prey is likely to 
lead to a better hunting strategy and diminish the mortality among Coyotes of all ages.

\paragraph{The data}
Warner Bros have over a period from 1949 to 2010 measured accurately a variety of evasive maneuvers. High-throughput measurement technology yielded close tracking of the animals through various stages of evasion. In addition, the company has genotyped each animal with their EvasionChip, a 100,000 SNP chip. The SNPs were chosen by their proximity to genes known or suspected to affect Evasion behavior in the prey and can be corresponded between the two species.

\paragraph{The model}
I propose to build a probabilistic model of evasive tactics of the two species ...
I will model a target prey's position using a Hidden Markov Model. The hidden states will range over a pool of evasive tactics surveyed in a recent article\cite{Baskerville:2011}. Further, I will assume that given a choice of current evasive tactic and prey's current position, the subsequent time-slice position can be accurately predicted using  a detailed kinematic model of evasive maneuvers recently published in \cite{Mangey:2011}. The key assumption in our model is that preferences for transition between different evasive maneuvers are genetically determined and transition matrices can be associated with a small subset of SNPs.

\paragraph{Secret sauce, if any}
I will employ a Lupine prior on transition matrices prefers sparse transition matrices and captures dependencies of transition probabilities between maneuvers on maneuver-features. Using a procedure reminiscent of automated relevance determination, enables robust recovery of relevant maneuver-features and their effect on transition matrices. The resulting transition matrices can be robustly associated with SNPs in our datasets ...

\paragraph{How the model will be evaluated}
I will evaluate performance of this method by using a held out dataset collected by our collaborators in New Mexico ...
The metric we will use is the prediction error of a prey's location ... We will also perform bootstrap analysis to assess the sensitivity of our associations between evasion technique transitions and SNPs

\paragraph{Anticipated challenges}
It is possible that the two species employ different evasive maneuvers and a joint model might lead to blurring of the two behaviors.
We will perform model selection and compare the performance of joint model vs. separate models for the two datasets. Regardless of outcome, this will result in
 useful observations about the similarities and differences of the two species.

\paragraph{The promise}
Given the increasing costs in lupine health insurance, improvement in understanding prey evasive tactics could lead to reduction in hunt-related injuries and subsequently to significant reduction in health related costs.

\bibliographystyle{plain}
\bibliography{evasion}

\end{document}